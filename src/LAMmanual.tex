\chapter{Návod k programu LAM od Igora Bodláka }
	
	Program LAM navazuje na letité výzkumy v oblasti zkoumání tvaru broušených šperků. V Centru strojového vnímání na Elektrotechnické fakultě v Praze před lety vznikl program LADOK. Ten umožňuje vytvořit matematický model kamene a simulovat průlet světelného svazku kamenem. 
	
	LAM vznikl pro porovnání výsledků ze simulace s reálným experimentem. Nám se podařilo postavit a zkalibrovat měřicí soustavu pro experimenty s broušenými kameny a tuto soustavu zkalibrovat. 
	
	Použití LAMu je omezené. Předešlé výzkumy zkoumaly pouze jeden z nejjednodušších tvarů kamene známého jako čtverec. 
	
	První částí programu je nalezení korespondujících bodů z experimentu ke svazkům ze simulačního programu. Myšlenkou celé věci bylo nasvícení kamene pouze na část kamene. V ideálním případě se jednalo o jedinou facetu. Tímto postupem se značně zmenšil počet bodů pro korespondenci. Laser, který svítil do kamene byl pohyblivý a bylo potřeba nasnímat řadu snímků s různými pozicemi laseru. Laser tedy dopadal na různé facety kamene a z něj vycházely příslušné svazky. S posunutím laseru vznikaly jednotlivé snímky. Z těchto snímků potom bylo možné určit korespondující body s modelem. 
	
	Druhou částí byla optimalizace tvaru kamene. Broušené plochy(facety) kamene byly brány za ideální rovné plochy a optimalizovala se poloha facet. Poloha facety byla vázána pevným bodem, okolo kterého se mohla natáčet. Vzdálenost tohoto bodu zůstávala konstantní. Facetě byl tedy ubrán jeden stupeň volnosti. 
	
	Pro optimalizaci byla zvolena kriteriální funkce. Výstupem této funkce byla vzdálenost bodu na stínítku od bodu, který by měl vzniknout na stínítku podle matematického modelu. 
	
	My jsme se snažili tento program upravit, aby se dal použít i pro námi zkoumaný kámen tvaru šatonové růže. Přitom jsme změnili některé výpočty nebo přístupy změnily a o mnohé možnosti rozšířili. 
	
	Pokusím se tedy popsat úpravy LAMu a zároveň k němu napsat manuál, neboli návod, jak postupovat při optimalizaci kamene jiného tvaru, než byl používán v experimentech. Tím mám na mysli VIVA12.
	
\section{Provedené změny}

\section{Manuál k LAMu}


 \clearpage